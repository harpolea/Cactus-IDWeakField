% *======================================================================*
%  Cactus Thorn template for ThornGuide documentation
%  Author: Ian Kelley
%  Date: Sun Jun 02, 2002
%  $Header$
%
%  Thorn documentation in the latex file doc/documentation.tex
%  will be included in ThornGuides built with the Cactus make system.
%  The scripts employed by the make system automatically include
%  pages about variables, parameters and scheduling parsed from the
%  relevant thorn CCL files.
%
%  This template contains guidelines which help to assure that your
%  documentation will be correctly added to ThornGuides. More
%  information is available in the Cactus UsersGuide.
%
%  Guidelines:
%   - Do not change anything before the line
%       % START CACTUS THORNGUIDE",
%     except for filling in the title, author, date, etc. fields.
%        - Each of these fields should only be on ONE line.
%        - Author names should be separated with a \\ or a comma.
%   - You can define your own macros, but they must appear after
%     the START CACTUS THORNGUIDE line, and must not redefine standard
%     latex commands.
%   - To avoid name clashes with other thorns, 'labels', 'citations',
%     'references', and 'image' names should conform to the following
%     convention:
%       ARRANGEMENT_THORN_LABEL
%     For example, an image wave.eps in the arrangement CactusWave and
%     thorn WaveToyC should be renamed to CactusWave_WaveToyC_wave.eps
%   - Graphics should only be included using the graphicx package.
%     More specifically, with the "\includegraphics" command.  Do
%     not specify any graphic file extensions in your .tex file. This
%     will allow us to create a PDF version of the ThornGuide
%     via pdflatex.
%   - References should be included with the latex "\bibitem" command.
%   - Use \begin{abstract}...\end{abstract} instead of \abstract{...}
%   - Do not use \appendix, instead include any appendices you need as
%     standard sections.
%   - For the benefit of our Perl scripts, and for future extensions,
%     please use simple latex.
%
% *======================================================================*
%
% Example of including a graphic image:
%    \begin{figure}[ht]
% 	\begin{center}
%    	   \includegraphics[width=6cm]{MyArrangement_MyThorn_MyFigure}
% 	\end{center}
% 	\caption{Illustration of this and that}
% 	\label{MyArrangement_MyThorn_MyLabel}
%    \end{figure}
%
% Example of using a label:
%   \label{MyArrangement_MyThorn_MyLabel}
%
% Example of a citation:
%    \cite{MyArrangement_MyThorn_Author99}
%
% Example of including a reference
%   \bibitem{MyArrangement_MyThorn_Author99}
%   {J. Author, {\em The Title of the Book, Journal, or periodical}, 1 (1999),
%   1--16. {\tt http://www.nowhere.com/}}
%
% *======================================================================*

% If you are using CVS use this line to give version information
% $Header$

\documentclass{article}

% Use the Cactus ThornGuide style file
% (Automatically used from Cactus distribution, if you have a
%  thorn without the Cactus Flesh download this from the Cactus
%  homepage at www.cactuscode.org)
\usepackage{../../../../doc/latex/cactus}

\begin{document}

% The author of the documentation
\author{Alice Harpole \textless alice.harpole@soton.ac.uk\textgreater \\ Ian Hawke \textless I.Hawke@soton.ac.uk\textgreater}

% The title of the document (not necessarily the name of the Thorn)
\title{IDWeakField}

% the date your document was last changed, if your document is in CVS,
% please use:
%    \date{$ $Date$ $}
\date{October 13 2015}

\maketitle

% Do not delete next line
% START CACTUS THORNGUIDE

% Add all definitions used in this documentation here
%   \def\mydef etc

% Add an abstract for this thorn's documentation
\begin{abstract}
    This thorn creates the initial data for a region of fluid in a weak gravitational field, given by the metric
        \begin{equation}
            ds^2 = -\left(1-\frac{2GM}{rc^2}\right)dt^2 + \left(1+\frac{2GM}{rc^2}\right)\left(dx^2 + dr^2\right),
        \end{equation}
    where \(G\) is the gravitational constant and \(M\) the mass of the star. The region can be set up such that the fluid contains a convectively unstable bubble, or two layers of fluid unstable to either the Rayleigh-Taylor or Kelvin-Helmholtz instabilities.
\end{abstract}

% The following sections are suggestive only.
% Remove them or add your own.

\section{Introduction}

\section{Bubble}
\label{sec:Bubble}
A circular region of underdense material was placed in a fluid vertically stratified by the gravitational field. The bubble is convectively unstable, so rises. The bottom boundary was reflective, the top outflow and the horizontal edges periodic.

\section{Kelvin-Helmholtz instability}
\label{sec:Kelvin-Helmholtz}
The Kelvin-Helmholtz instability was simulated by placing a dense fluid on top of less dense one, with gravitational field acting in the vertical direction. The two sections of fluid move slowly in opposite directions so there is shearing at their interface. The initial conditions for the horizontal velocity were:
\begin{equation}
U^x = \begin{cases} U_1 - U_m\exp\left(\frac{y - y_\text{center}}{L_x}\right) & \text{if } y \leq y_\text{center}\\
U_2 + U_m\exp\left(\frac{-y + y_\text{center}}{L_x}\right) & \text{if } y > y_\text{center}\end{cases},
\end{equation}
where \(L_x\) is the horizontal extent of the numerical domain, \(U_1\), \(U_2\) are the velocities of the sections fluid at infinity in the vertical direction and \(U_m = \frac{1}{2}(U_2-U_1)\). The initial conditions for the density are identical, but with \(D\) exchanged for \(U\).

Outflow boundaries were used for the vertical edges and periodic for the horizontal. Seed instability by making a small sinusoidal perturbation to the vertical velocity:
\begin{equation}
U^y = A \sin\left(\frac{4 \pi x}{L_x}\right),
\end{equation}
where \(A\) is some small constant (0.005 was used here).

\section{Rayleigh-Taylor instability}
\label{sec:Rayleigh-Taylor}
The Rayleigh-Taylor instability was modelled by playing a dense fluid on top of a less dense one, with a gravitational field acting in the vertical direction. The instability was then seeded by making small sinusoidal perturbation of the vertical velocity:
\begin{equation}
U^y = A \cos\left(\frac{2\pi x}{L_x}\right) \exp\left(-\frac{(y-y_\text{center})^2}{\sigma^2}\right),
\end{equation}
where \(L_x\) is the horizontal extent of the numerical domain, \(y_\text{center}\) is the vertical coordinate at the centre of the domain and \(A\), \(\sigma\) are some small constants (0.1 was used here). The boundary between the fluids of different densities was smoothed to eradicate very high density gradients in the initial data, which can be difficult to deal with numerically
\begin{equation}
D = D_1 + \frac{1}{2}(D_2 - D_1)\left(1 + \tanh\left[\frac{y - y_\text{center}}{0.5y_\text{smooth}}\right]\right),
\end{equation}
where \(D_1 < D_2\).
Outflow boundary conditions were used for the top and bottom boundaries and periodic for the horizontal boundaries. 

\section{Numerical Implementation}

\section{Using This Thorn}

\subsection{Obtaining This Thorn}

\subsection{Basic Usage}

\subsection{Special Behaviour}

\subsection{Interaction With Other Thorns}

\subsection{Examples}

\subsection{Support and Feedback}

\section{History}

\subsection{Thorn Source Code}

\subsection{Thorn Documentation}

\subsection{Acknowledgements}


\begin{thebibliography}{9}

\end{thebibliography}

% Do not delete next line
% END CACTUS THORNGUIDE

\end{document}
